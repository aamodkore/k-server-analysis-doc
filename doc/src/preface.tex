\tableofcontents

\chapter*{}
\addcontentsline{toc}{chapter}{Abstract}
\begin{center}
\textbf{Abstract}
\end{center}
The $k$-server problem, is the problem to move $k$ servers through a metric space serving a sequence of requests, making decisions in an online manner. It is perhaps the most influential online problem. It is a generalization of the paging problem and the weighted cache problem, which have widespread practical applications. The $k$-server conjecture, posed over two and a half decades ago, is still a open problem and is a major driving force in developing online algorithms. \\
We study a specific case of the server problem -- the 3-server problem on the cycle. We analyse the performance of online algorithms, specifically the work function algorithm, on the problem as compared to the optimal solution. For this we implement both the work function algorithm and the (offline) algorithm that gives the optimal solution. We develop different strategies for generating request sequences, in patterns that would provide insight on the general $k$-server conjecture. We also take a look at other parameters of the algorithm and try to formulate results based on the same.
