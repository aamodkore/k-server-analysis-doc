\chapter{Implementation, Tests and Results}\label{ch:tests-results}

\section{Implementation of Algorithms}\label{sec:algorithm-implementation}

The implementation of the algorithms and test cases is done in Python 3.2.3. In the sections that follow $k$ represents the number of servers (mostly and unless specified otherwise we consider $k=3$), $p$ represents the number of points on the cycle, $n$ represents the length of the request sequence and the approximation ratio is represented as $\alpha = \frac{\sf{COST}_{\sf{WFA}} (C_0,r)}{\mathcal{O}(C_0,r)}$.

\subsection{Optimal Algorithm}

We implemented the optimal offline algorithm (Section ~\ref{sec:optimal-algorithm}). For this we had to implement an algorithm to evaluate the  minimum cost maximum flow of an directed acyclic network. For the minimum cost maximum flow problem, we use minimum cost augmentation~\cite{Tar-b} using successive shortest paths~\cite{AMO-b} based on the cost of edges. The flow along the minimum cost path is successively augmented to the flow of the network and the residual network is again iteratively augmented, until no path from the source to the sink remains.

Since edges may have negative costs, we cannot use Dijkstra's algorithm. We need to find shortest path in a network where edges have negative weights, which calls for use of the Bellman-Ford algorithm. However, Bellman-Ford algorithm would take worst case $O((k+n)^2n)$ time complexity, for $k$ servers and a sequence of $n$ requests. Instead we establish node potentials and reduced  costs for edges using Bellman-Ford algorithm. Since reduced costs are non-negative and do not change shortest paths between nodes, we use Dijkstra's algorithm for finding shortest path. After augmenting the flow along the found path, the node potentials and reduced costs need to be updated.

\texttt{NetworkFlow.py} contains the implementation for the minimum cost maximum flow algorithm. The \texttt{FlowNetwork} class creates the directed acyclic network. Functions \texttt{add\_vertex} and \texttt{add\_edge} help build the required network. The function \texttt{max\_flow} and \texttt{min\_cost\_max\_flow} find the maximum flow and the maximum flow with minimum cost respectively. \\

Then, we create a directed acyclic network based on the number of servers and request sequence and compute the optimal service strategy based on the maximum flow with minimum cost of the created network. The \texttt{ServerSpace} class is created for these functionalities. It takes input the initial configuration of the servers and the distance function of the metric. A sequence of requests can then be added and the function \texttt{process\_requests} processes these requests and produces the output strategy.

\subsection{Work Function Algorithm}

The work function algorithm~\cite{KP94} (desribed in Section~\ref{sec:work-function-algorithm}) is implemented in \texttt{WorkFunction.py}. We use the dynamic programming approach to find the appropriate server to service the oncoming request. The already computed work function values are stored in the table \texttt{stored}, which is a dictionary type data structure. The dictionary data structure ensures fast $O(1)$ access to associations and also makes sure that rows in the table and elements in the rows are created only when needed. Thus, space in memory is occupied by only values of the function that are computed (i.e. there are no empty cells), giving efficient usage of memory.

While an object of the \texttt{WorkFunction} class is created, it takes as input the distance function of the metric space and the initial configuration $C_0$ of the servers. Further, \texttt{add\_request} and \texttt{delete\_request} can be used to add or remove requests to or from the end of the request sequence. The function \texttt{value} returns the value of the work function given the appropriate arguments. \texttt{process\_request} takes an integer $i$ and processes the request at the $i$th position in the sequence based on the work function algorithm and produces the server that serves the request and updates the configuration of the system to after that of the request has been served.

\section{The Cycle Metric Space}\label{sec:cycle-metric-implementation}

For all tests we considered a cycle with small number of points (mostly $p=20$ or $p=36$). Implementing the cycle as a metric space was simple. For the cycle with $p$ points, each point was taken as an integer from $0$ to $p-1$. The distance between two points in the metric is calculated based on their difference modulo $p$. If the difference (modulo $p$) is greater than $p/2$ we take the shorter distance along the other direction.

\lstset{language=Python}          % Set your language (you can change the language for each code-block optionally)

\begin{lstlisting}[caption=Code for distance function of a cycle with $p$ points , frame=single]  % Start your code-block

def cycle_metric(a,b) :
	d = (b-a)%p
	if d > p//2 :
		return (p-d)	
	else :
		return d
\end{lstlisting}



\section{Generating Requests}\label{sec:generating-requests}

The request sequence to be generated for the problem holds an important place. A good strategy to produce subsequent requests could have potential ways to attack the general $k$-server conjecture. 

\subsection{Random Generation}
An initial strategy (tried out mostly to check the implementation) was to generate requests randomly at one of the points in the metric. This did not produce very interesting results. 

As the sequence was random, both the work function algorithm and optimal strategy give similar performance. The ratio $\alpha = \frac{\sf{COST}_{\sf{WFA}} (C_0,r)}{\mathcal{O}(C_0,r)}$ (henceforth referred to as the approximation ratio of the algorrithm is quite close to $1.0$ (See Table~\ref{tab:random-approx-ratio}).

\begin{table}[!ht]
\begin{center}
\begin{tabular}{|c|c|c|c|c|c|}
\hline
$p$ & $n$ & $\sf{COST}_{\sf{WFA}} (C_0,r)$ & $\mathcal{O}(C_0,r)$ & $\alpha$ \\
\hline
20 & 100 & 202 & 174 & 1.161 \\
20 & 100 & 184 & 169 & 1.089 \\
20 & 100 & 186 & 152 & 1.224 \\
20 & 100 & 191 & 173 & 1.104 \\
20 & 100 & 211 & 189 & 1.116 \\
36 & 100 & 380 & 330 & 1.152 \\
36 & 100 & 362 & 306 & 1.183 \\
36 & 100 & 370 & 337 & 1.098 \\
36 & 100 & 320 & 294 & 1.088 \\
36 & 100 & 339 & 300 & 1.130 \\
\hline
\end{tabular}
\caption{Approximation Ratio of WFA for Random Generation}
\label{tab:random-approx-ratio}
\end{center}
\end{table}

\subsection{Generation on the Longest Arc}

We consider only the case of $k=3$. The $3$ servers divide the cycle into $3$ arcs. For a given configuration, we take the longest of the $3$ arcs and place the new requests at the mid-point of the arc (i.e. point equidistant from both servers at the end of the arc). If two points are in the mid of the arc, we randomly choose one of the two.

Table~\ref{tab:mid-approx-ratio} represents the values for the approximation ratio of the work function algorithm for this request generation strategy. For more data of the tests refer Section~\ref{sec:results}.

\begin{table}[!ht]
\begin{center}
\begin{tabular}{|c|c|c|c|c|c|}
\hline
$p$ & $n$ & $\sf{COST}_{\sf{WFA}} (C_0,r)$ & $\mathcal{O}(C_0,r)$ & $\alpha$ \\
\hline
20 & 100 & 500 & 174 &	2.874   \\
20  & 100 & 497       & 171	& 2.906 \\
20  & 100 & 499       & 171	& 2.918 \\
20  & 50  & 248       & 89	& 2.787  \\
20  & 50  & 253       & 87	& 2.908  \\
20  & 50  & 253       & 90	& 2.811  \\
20  & 50  & 252       & 88	& 2.864  \\
36  & 100 & 896       & 310	& 2.890 \\
36  & 100 & 899       & 313	& 2.872 \\
36  & 100 & 901       & 313	& 2.879 \\
36  & 50  & 450       & 163	& 2.761 \\
36  & 50  & 451       & 154	& 2.929 \\
36  & 50  & 453       & 167	& 2.713 \\
36  & 50  & 449       & 154	& 2.916 \\
\hline
\end{tabular}
\caption{Approximation Ratio of WFA for Generation on Longest Arc}
\label{tab:mid-approx-ratio}
\end{center}
\end{table}

\subsection{Worst Cost Generation}

The worst cost generation generates the request for which the immediate cost for processing the request based on the work function algorithm would be maximum. 
$$ r_{t+1} = \argmax_{r\in \mathcal{C}_p} (\sf{COST}_{\sf{WFA}} (C_0,(r_1,\hdots,r_t,r)) - \sf{COST}_{\sf{WFA}} (C_0,(r_1,\hdots,r_t)) $$

As it turns out worst cost generation is much similar to generartion in the middle of the longest arc. Thus it produces very similar results to the previous request sequence generation strategy. (See Section~\ref{sec:results}, Table~\ref{tab:worst-approx-ratio}). For both these strategies, the ratio is very close to $3.0$ but always less than $3.0$.

\subsection{Repeated Shifting}\label{subsec:repeated-shifting}

Repeated shifting involves shifting requests between adjacent locations until both of them are occupied by servers. We generate the request in phases. In each phase we take the longest arc formed by the server configuration and take the mid-point of the arc and one adjacent location to this mid-point. Then we alternately generate requests at these $2$ locations until servers occupy both these locations. Then we move on to the next phase. The following code gives an easy implementation of this strategy.

\begin{lstlisting}[caption=Code for single phase of repeated shifting , frame=single]  % Start your code-block

	mid = generate_middle(longest_arc(config))
	next = (mid+1)%p
	while True :
		if mid not in config :
			add_request(mid)
		elif next not in config :
			add_request(next)
		else :
			break
		i = i+1

\end{lstlisting}



\section{Evaluating Performance}\label{sec:evaluating-performance}

For evaluating performance further (specifically for the repeated shifting strategy), we also consider some other parameters. \\

We call a configuration $C_i$ \bif{terminal} if a particular server in the configuration does not move after this configuration. Formally, we say that $C_i$ is a terminal configuration if there exists $s\in\{1,\hdots,k\}$ such that, for all $t>i$, $C_t[s] = C_i[s]$, where $C_i[j]$ represents the $j$th element (i.e. position of the $j$th server) in the ordered set configuration. $C_i$.

Also, for a point on the cycle $r$, the point $\overline{r}$, as described in Section~\ref{cycle-metric-space}, represents the diametrically opposite point on the cycle.

We are particularly interested in the following performance parameters--
\begin{itemize}
\item
$\mathcal{D} = \sum_{i=1}^{n} (w_i(C_{i-1}) - w_{i-1}(C_{i-1}))$
\item
$\mathcal{A} = \sum_{i=1}^{n} (w_i(\overline{r_i}^k) - w_{i-1}(\overline{r_i}^k))$
\item
$\mathcal{T} = \sum_{C_i\text{ is terminal}} w_i(C_i)$
\end{itemize}

\begin{table}[!ht]
\begin{center}
\begin{tabular}{|c|c|c|c|c|}
\hline
$p$ & $n$ & $\overline{\mathcal{D}}$ & $\overline{\mathcal{A}}$ & $\overline{\mathcal{T}}$ \\
\hline
20 & 100 & 260 & 270 & 213  \\
20 & 200 & 520 & 532 & 454  \\
20 & 250 & 648 & 658 & 574  \\
20 & 500 & 1270 & 1367 & 1151  \\
20 & 1000 & 2576 & 2588 & 2386  \\
20 & 2000 & 5056 & 5077 & 4628  \\
20 & 5000 & 12735 & 12772 & 11776  \\
20 & 10000 & 25147 & 25158 & 23308  \\
36 & 100 & 281 & 299 & 201  \\
36 & 200 & 558 & 572 & 437  \\
36 & 500 & 1343 & 1362 & 1120  \\
36 & 1000 & 2654 & 2675 & 2271  \\
36 & 2000 & 5292 & 5332 & 4577  \\
36 & 5000 & 13093 & 13170 & 11315  \\
36 & 10000 & 26301 & 26399 & 22921  \\
\hline
\end{tabular}
\caption{Performance Parameters for Repeated Shifting (Mean Data, for individual case data see Section~\ref{sec:results})}
\label{tab:avg-stats-repeated-shift}
\end{center}
\end{table}

For all test cases (Table~\ref{tab:avg-stats-repeated-shift}), it is observed that, $\mathcal{T} < \mathcal{D} < \mathcal{A} $. Thus, 
$$\sum_{C_i\text{ is terminal}} w_i(C_i) < \sum_{i=1}^{n} (w_i(C_{i-1}) - w_{i-1}(C_{i-1})) < \sum_{i=1}^{n} (w_i(\overline{r_i}^k) - w_{i-1}(\overline{r_i}^k))$$

\section{Test Results}\label{sec:results}

This section covers the major aspects of the results. However, this is not the exhaustive cover of all the values obtained during the testing. The complete set of results can be found along with the implementation (in the \texttt{data} folder).

\begin{table}[!hb]
\parbox{.45\linewidth}{
\centering
\begin{tabular}{|c|c|c|c|c|c|}
\hline
$p$ & $n$ & $\sf{COST}_{\sf{WFA}}$ & $\mathcal{O}$ & $\alpha$ \\
\hline
20  & 100  & 202       & 174	& 1.161 \\
20  & 100  & 184       & 169	& 1.089 \\
20  & 100  & 186       & 152	& 1.224 \\
20  & 100  & 191       & 173	& 1.104 \\
20  & 100  & 211       & 189	& 1.116 \\
20  & 100  & 190       & 178	& 1.067 \\
20  & 100  & 177       & 165	& 1.073 \\
20  & 100  & 198       & 162	& 1.222 \\
20  & 100  & 211       & 179	& 1.179 \\
20  & 100  & 194       & 170	& 1.141 \\
36  & 100  & 391       & 313	& 1.249 \\
36  & 100  & 396       & 328	& 1.207 \\
36  & 100  & 337       & 308	& 1.094 \\
36  & 100  & 402       & 350	& 1.149 \\
36  & 100  & 397       & 313	& 1.268 \\
36  & 100  & 380       & 330	& 1.152 \\
36  & 100  & 362       & 306	& 1.183 \\
36  & 100  & 370       & 337	& 1.098 \\
36  & 100  & 320       & 294	& 1.088 \\
36  & 100  & 339       & 300	& 1.130 \\
90  & 50   & 423       & 363	& 1.165 \\
90  & 50   & 418       & 343	& 1.219 \\
90  & 50   & 403       & 389	& 1.036 \\
90  & 50   & 473       & 403	& 1.174 \\
90  & 50   & 388       & 388	& 1.000 \\
90  & 50   & 501       & 419	& 1.196 \\
90  & 50   & 425       & 379	& 1.121 \\
90  & 50   & 481       & 393	& 1.224 \\
90  & 50   & 404       & 374	& 1.080 \\
90  & 50   & 449       & 387	& 1.160 \\
\hline
\end{tabular}
}
\hfill
\parbox{.45\linewidth}{
\centering
\begin{tabular}{|c|c|c|c|c|c|}
\hline
$p$ & $n$ & $\sf{COST}_{\sf{WFA}}$ & $\mathcal{O}$ & $\alpha$ \\
\hline
90  & 100  & 832       & 794	& 1.048 \\
90  & 100  & 872       & 796	& 1.095 \\
90  & 100  & 1010      & 834	& 1.211 \\
90  & 100  & 855       & 751	& 1.138 \\
90  & 100  & 866       & 786	& 1.102 \\
90  & 100  & 991       & 843	& 1.176 \\
90  & 100  & 861       & 737	& 1.168 \\
90  & 100  & 848       & 788	& 1.076 \\
90  & 100  & 913       & 755	& 1.209 \\
90  & 100  & 926       & 804	& 1.152 \\
100 & 50  & 440  & 390	& 1.128 \\
100 & 50  & 381  & 375	& 1.016 \\
100 & 50  & 539  & 453	& 1.190 \\
100 & 50  & 407  & 386	& 1.054 \\
100 & 50  & 444  & 432	& 1.028 \\
100 & 50  & 532  & 452	& 1.177 \\
100 & 50  & 523  & 455	& 1.149 \\
100 & 50  & 537  & 442	& 1.215 \\
100 & 50  & 523  & 453	& 1.155 \\
100 & 50  & 549  & 455	& 1.207 \\
100 & 100 & 900  & 840	& 1.071 \\
100 & 100 & 980  & 860	& 1.140 \\
100 & 100 & 1155 & 907	& 1.273 \\
100 & 100 & 1022 & 870	& 1.175 \\
100 & 100 & 1017 & 867	& 1.173 \\
100 & 100 & 953  & 907	& 1.051 \\
100 & 100 & 1060 & 910	& 1.165 \\
100 & 100 & 1123 & 947	& 1.186 \\
100 & 100 & 1075 & 949	& 1.133 \\
100 & 100 & 1048 & 933	& 1.123 \\
\hline
\end{tabular}
}
\caption{Data for Random Generation for Large Number of Points}
\label{tab:dat-20}
\end{table}


\begin{table}[!ht]
\parbox{.45\linewidth}{
\centering
\begin{tabular}{|c|c|c|c|c|c|}
\hline
$p$ & $n$ & $\sf{COST}_{\sf{WFA}}$ & $\mathcal{O}$ & $\alpha$ \\
\hline
100 & 100 & 3080 & 1277	& 2.412     \\
100 & 100 & 3088 & 1279	& 2.414     \\
100 & 100 & 3092 & 1294	& 2.389     \\
100 & 100 & 3092 & 1305	& 2.369     \\
100 & 100 & 3103 & 1309	& 2.371     \\
100 & 100 & 3104 & 1311	& 2.368     \\
100 & 100 & 3109 & 1289	& 2.412     \\
100 & 100 & 3110 & 1312	& 2.370     \\
100 & 100 & 3110 & 1317	& 2.361     \\
100 & 100 & 3115 & 1310	& 2.378     \\
100 & 200 & 6146 & 2577	& 2.385     \\
100 & 200 & 6171 & 2561	& 2.410     \\
100 & 200 & 6198 & 2599	& 2.385     \\
100 & 200 & 6199 & 2594	& 2.390     \\
100 & 200 & 6202 & 2597	& 2.388     \\
100 & 50  & 1543 & 653	& 2.363      \\
100 & 50  & 1553 & 670	& 2.318      \\
100 & 50  & 1555 & 656	& 2.370      \\
100 & 50  & 1557 & 638	& 2.440      \\
100 & 50  & 1559 & 664	& 2.348      \\
100 & 50  & 1560 & 653	& 2.389      \\
100 & 50  & 1561 & 652	& 2.394      \\
100 & 50  & 1564 & 654	& 2.391      \\
100 & 50  & 1565 & 665	& 2.353      \\
100 & 50  & 1568 & 636	& 2.465      \\
100 & 75  & 2311 & 967	& 2.390      \\
100 & 75  & 2317 & 968	& 2.394      \\
100 & 75  & 2319 & 964	& 2.406      \\
100 & 75  & 2323 & 972	& 2.390      \\
100 & 75  & 2325 & 988	& 2.353      \\
100 & 75  & 2326 & 979	& 2.376      \\
100 & 75  & 2330 & 974	& 2.392      \\
100 & 75  & 2333 & 985	& 2.369      \\
100 & 75  & 2335 & 997	& 2.342      \\
100 & 75  & 2340 & 984	& 2.378      \\
\hline
\end{tabular}
}
\hfill
\parbox{.45\linewidth}{
\centering
\begin{tabular}{|c|c|c|c|c|c|}
\hline
$p$ & $n$ & $\sf{COST}_{\sf{WFA}}$ & $\mathcal{O}$ & $\alpha$ \\
\hline
90  & 100  & 2768       & 1153	& 2.401 \\
90  & 100  & 2771       & 1160	& 2.389 \\
90  & 100  & 2772       & 1123	& 2.468 \\
90  & 100  & 2782       & 1160	& 2.398 \\
90  & 100  & 2790       & 1161	& 2.403 \\
90  & 100  & 2812       & 1184	& 2.375 \\
90  & 100  & 2816       & 1196	& 2.355 \\
90  & 100  & 2819       & 1186	& 2.377 \\
90  & 100  & 2821       & 1190	& 2.371 \\
90  & 100  & 2825       & 1188	& 2.378 \\
90  & 200  & 5511       & 2296	& 2.400 \\
90  & 200  & 5622       & 2376	& 2.366 \\
90  & 200  & 5627       & 2376	& 2.368 \\
90  & 200  & 5628       & 2371	& 2.374 \\
90  & 200  & 5638       & 2391	& 2.358 \\
90  & 50   & 1384       & 578	& 2.394  \\
90  & 50   & 1385       & 585	& 2.368  \\
90  & 50   & 1385       & 590	& 2.347  \\
90  & 50   & 1390       & 572	& 2.430  \\
90  & 50   & 1407       & 591	& 2.381  \\
90  & 50   & 1411       & 594	& 2.375  \\
90  & 50   & 1414       & 615	& 2.299  \\
90  & 50   & 1415       & 591	& 2.394  \\
90  & 50   & 1416       & 604	& 2.344  \\
90  & 50   & 1423       & 606	& 2.348  \\
90  & 75   & 2076       & 873	& 2.378  \\
90  & 75   & 2083       & 871	& 2.392  \\
90  & 75   & 2089       & 873	& 2.393  \\
90  & 75   & 2099       & 883	& 2.377  \\
90  & 75   & 2105       & 893	& 2.357  \\
90  & 75   & 2111       & 903	& 2.338  \\
90  & 75   & 2114       & 915	& 2.310  \\
90  & 75   & 2117       & 893	& 2.371  \\
90  & 75   & 2120       & 915	& 2.317  \\
90  & 75   & 2121       & 899	& 2.359  \\
\hline
\end{tabular}
}
\caption{Data for Worst Cost Generation for Large Number of Points}
\label{tab:worst-approx-ratio}
\end{table}


\begin{table}[!htb]
\parbox{.45\linewidth}{
\centering
\begin{tabular}{|c|c|c|c|}
\hline
Case No. & $\mathcal{D}$ & $\mathcal{A}$ & $\mathcal{T}$ \\
\hline
1   & 250   & 259   & 205   \\
2   & 269   & 278   & 212   \\
3   & 250   & 259   & 205   \\
4   & 250   & 259   & 205   \\
5   & 254   & 261   & 203   \\
6   & 259   & 270   & 214   \\
7   & 247   & 259   & 203   \\
8   & 247   & 259   & 203   \\
9   & 260   & 270   & 214   \\
10  & 254   & 261   & 203   \\
11  & 247   & 262   & 203   \\
12  & 260   & 272   & 214   \\
13  & 250   & 264   & 209   \\
14  & 263   & 281   & 213   \\
15  & 255   & 264   & 206   \\
16  & 266   & 277   & 214   \\
17  & 284   & 293   & 238   \\
18  & 285   & 293   & 238   \\
19  & 284   & 294   & 241   \\
20  & 261   & 270   & 216   \\
\hline
\end{tabular}
\caption{Repeated Shifting ($p=20$, $n=100$)}
\label{tab:dat-20}
}
\hfill
\parbox{.45\linewidth}{
\centering
\begin{tabular}{|c|c|c|c|}
\hline
Case No. & $\mathcal{D}$ & $\mathcal{A}$ & $\mathcal{T}$ \\
\hline
21 & 520 & 531 & 464 \\
22 & 498 & 511 & 436 \\
23 & 530 & 542 & 469 \\
24 & 520 & 530 & 442 \\
25 & 529 & 539 & 468 \\
26 & 518 & 533 & 464 \\
27 & 502 & 509 & 436 \\
28 & 514 & 536 & 436 \\
29 & 529 & 542 & 462 \\
30 & 495 & 505 & 433 \\
31 & 510 & 522 & 440 \\
32 & 527 & 539 & 466 \\
33 & 531 & 542 & 467 \\
34 & 522 & 535 & 460 \\
35 & 518 & 526 & 442 \\
36 & 530 & 539 & 467 \\
37 & 505 & 518 & 436 \\
38 & 522 & 543 & 441 \\
39 & 550 & 564 & 492 \\
40 & 535 & 543 & 468 \\
\hline
\end{tabular}
\caption{Repeated Shifting ($p=20$, $n=200$)}
\label{tab:dat-20}
}
\end{table}


\begin{table}[!hb]
\parbox{.45\linewidth}{
\centering
\begin{tabular}{|c|c|c|c|}
\hline
Case No. & $\mathcal{D}$ & $\mathcal{A}$ & $\mathcal{T}$ \\
\hline
111 & 12687 & 12701 & 12047 \\
112 & 12489 & 12761 & 11100 \\
113 & 12702 & 12711 & 11526 \\
114 & 12714 & 12728 & 12074 \\
115 & 12698 & 12708 & 11527 \\
116 & 12688 & 12701 & 12052 \\
117 & 12729 & 12740 & 11555 \\
118 & 12335 & 12345 & 11491 \\
\hline
\end{tabular}
\caption{Repeated Shifting ($p=20$, $n=5000$)}
\label{tab:dat-20}
}
\hfill
\parbox{.45\linewidth}{
\centering
\begin{tabular}{|c|c|c|c|}
\hline
Case No. & $\mathcal{D}$ & $\mathcal{A}$ & $\mathcal{T}$ \\
\hline
121 & 24669 & 24679 & 23006 \\
122 & 25427 & 25437 & 23097 \\
123 & 24667 & 24678 & 23006 \\
124 & 26290 & 26300 & 24406 \\
125 & 24686 & 24698 & 23029 \\
\hline
\end{tabular}
\caption{Repeated Shifting ($p=20$, $n=10000$)}
\label{tab:dat-20}
}
\end{table}



\begin{table}[!htb]
\parbox{.45\linewidth}{
\centering
\begin{tabular}{|c|c|c|c|}
\hline
Case No. & $\mathcal{D}$ & $\mathcal{A}$ & $\mathcal{T}$ \\
\hline
126 & 157 & 171 & 83  \\
127 & 148 & 165 & 86  \\
128 & 139 & 158 & 81  \\
129 & 132 & 150 & 76  \\
130 & 151 & 167 & 87  \\
131 & 138 & 157 & 78  \\
132 & 162 & 174 & 84  \\
133 & 135 & 157 & 79  \\
134 & 158 & 171 & 85  \\
135 & 187 & 206 & 124 \\
136 & 139 & 157 & 78  \\
137 & 162 & 173 & 85  \\
138 & 151 & 165 & 80  \\
139 & 141 & 161 & 81  \\
140 & 140 & 156 & 80  \\
141 & 148 & 163 & 82  \\
142 & 154 & 167 & 83  \\
143 & 151 & 168 & 88  \\
144 & 156 & 171 & 84  \\
145 & 147 & 164 & 80  \\
\hline
\end{tabular}
\caption{Repeated Shifting ($p=36$, $n=100$)}
\label{tab:dat-20}
}
\hfill
\parbox{.45\linewidth}{
\centering
\begin{tabular}{|c|c|c|c|}
\hline
Case No. & $\mathcal{D}$ & $\mathcal{A}$ & $\mathcal{T}$ \\
\hline
146 & 276 & 293 & 193 \\
147 & 305 & 328 & 230 \\
148 & 262 & 280 & 187 \\
149 & 266 & 284 & 187 \\
150 & 271 & 287 & 189 \\
151 & 279 & 294 & 193 \\
152 & 317 & 336 & 238 \\
153 & 307 & 330 & 230 \\
154 & 274 & 287 & 189 \\
155 & 275 & 290 & 193 \\
156 & 317 & 338 & 238 \\
157 & 266 & 284 & 188 \\
158 & 274 & 287 & 191 \\
159 & 265 & 284 & 190 \\
160 & 315 & 333 & 235 \\
161 & 262 & 280 & 187 \\
162 & 264 & 280 & 187 \\
163 & 277 & 295 & 199 \\
164 & 268 & 282 & 185 \\
165 & 281 & 299 & 196 \\
\hline
\end{tabular}
\caption{Repeated Shifting ($p=36$, $n=200$)}
\label{tab:dat-20}
}
\end{table}


\begin{table}[!hb]
\parbox{.45\linewidth}{
\centering
\begin{tabular}{|c|c|c|c|}
\hline
Case No. & $\mathcal{D}$ & $\mathcal{A}$ & $\mathcal{T}$ \\
\hline
238 & 13063 & 13214 & 11209 \\
239 & 13200 & 13219 & 11577 \\
240 & 12932 & 13018 & 10993 \\
241 & 13062 & 13213 & 11215 \\
242 & 13061 & 13213 & 11209 \\
243 & 12942 & 13026 & 10996 \\
244 & 13145 & 13162 & 11530 \\
245 & 12946 & 13026 & 10993 \\
\hline
\end{tabular}
\caption{Repeated Shifting ($p=36$, $n=5000$)}
\label{tab:dat-20}
}
\hfill
\parbox{.45\linewidth}{
\centering
\begin{tabular}{|c|c|c|c|}
\hline
Case No. & $\mathcal{D}$ & $\mathcal{A}$ & $\mathcal{T}$ \\
\hline
246 & 26152 & 26443 & 22499 \\
247 & 26325 & 26342 & 23134 \\
248 & 26820 & 26838 & 23830 \\
249 & 25872 & 26018 & 22009 \\
250 & 26336 & 26354 & 23136 \\
\hline
\end{tabular}
\caption{Repeated Shifting ($p=36$, $n=10000$)}
\label{tab:dat-20}
}
\end{table}



