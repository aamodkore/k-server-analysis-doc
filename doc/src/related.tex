\section{Related Work}\label{sec:related-work}

The server problem was first defined by Manasses, McGeogh and Sleator~\cite{MMS88} in 1988. It was a special case of the online metrical task systems problem stated by Borodin et al.\ ~\cite{BLS87, BLS92} earlier. Manasse et al.\ showed few important results -- They showed that no online algorithm can have competitive ratio less than $k$, as long as the metric space has more than $k$ distinct points. Further, they showed that the competitive ratio is exactly $2$ for for the special case of $k=2$ and that it is exactly $k$ for all metric spaces with $k+1$ points. With this evidence, they posed the $k$-server conjecture (Conjecture~\ref{thrm:k-server-conjecture}).

Computer experiments on small metric spaces verified the conjecture for $k=3$. The conjecture was shown to hold for the line (1-dimensional Euclidean space)~\cite{CKPV91} and for tree metric spaces~\cite{CL91}. An optimal algorithm for the offline problem was also established~\cite{CKPV91}. In 1994, a dramatic improvement was shown by Koutaoupias et al.~\cite{KP94} which established the work function algorithm and showed that it has competitive ratio $2k-1$. This remains the best known bound~\cite{Kou09} and there has been limited progress on the server problem. In a recent survey~\cite{Kou09}, Koutsoupias analyses some major results about the problem, specially concerning the 1-dimensional Euclidian metric, tree metrics and metric spaces with $k+1$ points.

Two special cases of the server problem we are interested in are the $3$-server problem~\cite{CL94, BCL02} and the $k$-server problem on a cycle~\cite{FRRS91}. Any progress on these problems may lead to new paths to attack the $k$-server conjecture. For both these cases, nothing better than the $2k-1$ bound is known.